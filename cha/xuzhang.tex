\frontmatter
\thispagestyle{empty}
\newpage
\begin{center}
	\textbf{\LARGE 不那么重要的前言}
\end{center}
——献给曾经的高中技术部,书籍更新中
\begin{ascolorbox5}{第二版序}
	距离开始写这本书,掐指一算已经有 6 个月的时间了,这 6 个月的大学生活经历了太多,从一个健康高中生变成了一个脆皮大学生,感觉是对自己的自嘲,也是对现实的无奈。
	
	曾经,有很多的人问我,能不能做一些电脑基本操作的教程,但是我实话实说,我也觉得自己才疏学浅,至少没有到教别人的地步;然而,随着时间的推移,我开始意识到,每个人都有自己独特的学习经验和理解方式,或许我的经验对他人来说并不那么晦涩难懂。
	
	于是,我决定尝试将自己的学习心得和 Windows 11 基本操作经验分享出来,希望能够帮助那些刚刚踏入电脑世界的朋友更好地适应和利用这个数字化时代的工具。
	
	当初写的时候,一天大概得花不少时间去写这个东西,而且还挺累人的,查阅资料还要结合自身,然后是书中截图也要自己写。本书也参考了不少\LaTeX 的模板,包括盒子环境等美化排版的一些玩意,其实是想致力于打造一本\md{符合各位审美}的书籍。
	
	标题是 Windows 11 从入门到重新入门,其实个人认为很恰当,因为我感觉我查了各方资料,我也感觉像是自己重新入门了一遍。现如今有各方不擅长使用电脑的萌新,但是网络上面也没有详细的使用文档,不过微软官方倒是在开始菜单栏内运用了“提示”,各位观看文章之余,可以多去看看官方提示,有助于提升你的使用水平。
	
	另外,感谢各位的支持,本书完全开源免费,且任何人均可编辑,本书已经上传至 GitHub 供各路高手修改,审阅。当无数双眼睛看这本书时,书中错误也就无处可躲了。
	
	感谢读者!
\end{ascolorbox5}


\begin{flushright}
	匡睿同学 \\
	December 12, 2023
\end{flushright}

\begin{ascolorbox5}{第一版序}
	这本书是我查阅各方资料写出来的,查重率不高大可放心食用。
	
	似乎也写了好几天,一天大概得花不少时间去写这个东西,而且还挺累人的,查阅资料还要结合自身,然后是书中截图也要自己写。本书也参考了不少 LATEX 的模板,包括盒子环境等美化排版的一些玩意,其实是想致力于打造一本优雅的书籍(但是也没优雅到哪去)。
	
	标题是 Windows 11 从入门到重新入门,其实个人认为很恰当,因为我感觉我查了各方资料,我也感觉像是自己重新入门了一遍。现如今有各方不擅长使用电脑的萌新,但是网络上面也没有详细的使用文档,不过微软官方倒是在开始菜单栏内运用了“提示”,各位观看文章之余,可以多去看看官方提示,有助于提升你的使用水平。
	
	另外,感谢各位的支持,本书完全开源免费,且任何人均可编辑,本书稍后会上传至 GitHub 供各路高手修改,审阅。当无数双眼睛看这本书时,书中错误也就无处可躲了。
	
	感谢读者!
\end{ascolorbox5}

\begin{flushright}
	匡睿同学 \\
	October 12, 2023
\end{flushright}
%\newpage
%\begin{center}
%	\textbf{\LARGE 内容概述}
%\end{center}
%\begin{ascolorbox17}{内容概述}
%\begin{minipage}[b]{0.49\textwidth}
%	\begin{dinglist}{118}
%		\item 
%%		\item 001 网络爬虫初始
%%		\item 002 浏览器开发者工具使用
%%		\item 003 HTTP协议与HTTPS协议
%%		\item 004 Requests库基本使用
%%		\item 005 xpath与lxml模块使用
%%		\item 006 正则表达式与re模块使用
%%		\item 007 CSS选择器与BS4库使用
%%		\item 008 jquery与PyQuery模块使用
%%		\item 009 Json数据与Json模块的使用
%%		\item 010 数据存储Pandas使用
%	\end{dinglist}
%\end{minipage}
%\begin{minipage}[b]{0.49\textwidth}
%	\begin{dinglist}{118}
%		\item 
%%		\item 011 数据存储MySQL与MongoDb使用
%%		\item 012 爬虫进阶Selenium的使用
%%		\item 013 爬虫进阶多进程爬虫
%%		\item 014 爬虫进阶多线程爬虫
%%		\item 015 爬虫进阶多协程爬虫
%%		\item 016 爬虫进阶异步爬虫
%%		\item 017 爬虫进阶Scapy爬虫
%%		\item 018 爬虫进阶分布式爬虫
%%		\item 019 爬虫进阶APP爬虫
%%		\item 020 爬虫序章反爬
%	\end{dinglist}
%\end{minipage}
%\end{ascolorbox17}
\frontmatter
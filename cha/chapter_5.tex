\chapter{存储并整理文件}
\begin{center}
	\textcolor[RGB]{255, 0, 0}{\faHeart}请记住那些对你好的人,因为他们本可以不这样的.\textcolor[RGB]{255, 0, 0}{\faHeart}
\end{center}
\rightline{——《千与千寻》}
\vspace{-5pt}
\begin{center}
	\pgfornament[width=0.36\linewidth,color=lsp]{88}
\end{center}

\section{浏览文件资源管理器}
正如上图 5.1 所示,Windows 11 中并非一切都发生了变化。有些基本的东西依旧不变,除了上面新增了标签,可以在一个窗口上打开多个文件夹。

你可以将文件保存在文件资源管理器上打开的文件夹中。

\begin{figure}[h]
	\begin{center}
		\includegraphics[width=\textwidth]{figure/prtsc/menu_bar.png}
		\caption{文件资源管理器}
	\end{center}
\end{figure}

\subsection{了解文件夹}
文件夹是存储文件的位置。要访问它,只需左键双击,它就会打开。它们通常如图 5.2 所示:
\begin{figure}[h]
	\begin{center}
		\includegraphics{figure/prtsc/folders.jpg}
		\caption{文件夹}
	\end{center}
\end{figure}
文件夹使你的文件井井有条(那得看你怎么用)。要创建一个,请单击文件资源管理器左上角的“新建”。

\subsection{查看驱动器、文件夹和其他媒体}

\begin{definition}{驱动器}
	驱动器指的是磁盘驱动器。通过某个文件系统格式化并带有一个驱动器号的存储区域。存储区域可以是软盘、CD、硬盘或其他类型的磁盘。单击“Windows资源管理器”或“此电脑”中相应的图标可以查看驱动器的内容。
\end{definition}

当你打开驱动器时,您会看到其中的所有可用文件和媒体。

驱动器上的文件包括文件夹、媒体等以及重要的软件。在驱动器上写入可能需要一些\md{额外的权限},因为删除驱动器上的文件夹可能会对已安装的软件产生不利影响,特别是\md{系统盘}(通常为C盘)

\subsection{查看文件夹内的内容}
文件夹内包括其他文件夹、文件和媒体,可以命名和重命名他们以便轻松找到。

类似于驱动器,只不过驱动器不能放在任何文件夹内。

\subsection{重命名文件或文件夹}
要重命名文件或文件夹,请右键单击它并找到\md{“重命名”}。也可以\md{选中后}只需左键单击该图标一次或者按下键盘 F2 ,然后对其进行重命名。

\subsection{多选文件或文件夹}
要选择多个文件,可以使用鼠标按住屏幕的空白部分并将其拖动到要选择的文件上。若中途要跳过一个,可以按住 ctrl 键保持刚刚的一部分选中。

\subsection{删除文件或文件夹}
选中后按下键盘的 delete 键,或者右击删除甚至你可以直接移动到回收站图标中。

\subsection{复制或移动文件和文件夹}
若要复制到其他地方,则选中后右击选中\md{复制图标(图 5.3)}或使用组合键 Ctrl+C,在新的地方右击并选择\md{粘贴图标(图 5.4)}或使用组合键 Ctrl+V

\begin{figure}[h]
	\begin{center}
		\includegraphics{figure/prtsc/c.png}
		\caption{复制}
	\end{center}
\end{figure}

\begin{figure}[h]
	\begin{center}
		\includegraphics{figure/prtsc/v.png}
		\caption{粘贴}
	\end{center}
\end{figure}

若要移动,则是选中后使用如图 5.3 中带剪刀的图标,再到目标位置右击并选择\md{粘贴图标(图 5.4)}或使用组合键 Ctrl+V

\subsection{查看有关文件和文件夹的更多信息}
要查看有关文件和文件夹的更多信息,只需右键单击并转到\md{属性},或使用组合键 Alt+Enter 。在属性中,您将找到有关该文件夹的更多信息。此外,选择或将鼠标悬停在文件夹上将为您提供文件夹和文件的简要信息

\section{写入 CD 和 DVD}
刻录成 CD 和 DVD 是跨设备复制和共享文件的一种方式,因此 Windows 11 还可以刻录 CD 和 DVD。

现在市面上有两种常见的 CD (\md{现在谁还用 CD })它们是:CD-R和CD-RW,CD-R较便宜,但一旦写满就不能擦除重写。但是,您可以在 CD-RW 上重写。DVD 也是如此。 r 不可重写,而 RW 可以重写。

要复制文件,只需选择该文件并将其拖到您的首选目录即可。此外,你还可以右键单击该文件并选择复制或剪切。然后将其粘贴到您的首选目录。复制或剪切文件后,右键单击要将文件移动到的\md{目录中的空白区域}并将其粘贴到那里。

\begin{ascolorbox19}[可录光碟与可重复刻录光碟]{章外拓展}
	\md{可录光碟}(compact disc-recordable, CD-R)是一种可单次录写的只读记忆光碟。
	
	早期的可录光碟用银作涂层,容量只有650MB。后来推出保存期更长、用金作涂层的可录光碟。不过由于后来金的价格不断上涨,使生产商想尽办法制作涂层更薄的碟片。
	
	后来有机化学技术的进展,使可录光碟可以无需采用金或银作涂层,而且使较高容量的可录光碟(700MB)生产更划算。2009年在香港,一般牌子的可录光碟只需要大约1港元就可以买到,使可录光碟变得极为普遍。
	
	\md{可重复刻录光碟}(Compact Disc ReWritable;CD-RW)的记录层是一种相变合金,跟CD-R中的染色聚合体不同。在一片新的CD-RW中,相变合金以一种高反光性晶体形式存在,当红光高功率镭射光束对相变合金加热,被加热的部分便会变得黯淡无光,这就相当于一般光碟上的微坑。如果再次用红光高功率的镭射加热相变合金,它又会变回晶体,这样就可以再次写入资料,这无形中提升了存储备份的使用价值,但由于材料的特性关系,所以其改变次数有所限制,大约在一千次左右,以后期对镭射光的反射率只有15%,远低于CD-R的65%,在此情况下只有使用MultiRead功能的光驱才能正常读取。
\end{ascolorbox19}

\section{使用闪存驱动器和存储卡}


\begin{definition}{闪存驱动器}
	闪存驱动器(英语:USB flash drive)又称\md{随身碟},\md{U盘}。是一种使用USB协议连接计算机,通常通过闪存来进行数据存储的小型便携存储设备。一般U盘体积极小、重量轻、可重复写入,面世后迅速普及并取代传统的软盘及软盘驱动器。有时读卡器也会被归类为U盘,但这类设备的存储芯片并不是内置的,而是可以抽换的存储卡。[1]
\end{definition}
\footnotetext[1]{https://zh.wikipedia.org/wiki/OneDrive}
Windows 11 可与闪存驱动器和存储卡配合使用。连接后,可以在其中找到闪存驱动器或存储卡的名称。就像计算机上的其他文件夹或驱动器一样访问它们。

\section{OneDrive}
\begin{definition}{OneDrive}
	Microsoft OneDrive(旧称 Windows Live SkyDrive)是微软公司所推出的网络硬盘及云端服务。用户可以上传他们的文件到网络服务器上,并且透过网络浏览器来浏览那些文件。更可直接编辑和观看 Microsoft Office 文件。同时推出同步上载软件,可于电脑直接访问和同步文件。另外,OneDrive 并允许用户透过 Microsoft Account 来限制不同的用户访问文件,允许用户决定是否将文件与公众分享,或是限于联系菜单上的人才能访问;而对所有人公开的文件则不需要Microsoft Account即可访问。
\end{definition}
人话:微软给你的网盘。你可能听说过它或在计算机上尤其是现代计算机上见过它。Microsoft One Drive 是一种高效的工具,它使用云数据库系统来存储您的整个文档。无论是供您使用、办公室使用还是学校使用,都有很多计划(坑钱)可以存储你的整个有用文档。它们是一种方便的设备,非常有效且高效,可以安全可靠地处理计算机中的所有文档和文件。无论您走到哪里,只需登录一下即可获得保存的文件。

Office 365 订阅用户可获得 1TB 储存空间,免费用户则有 5GB 储存空间(用户除了免费空间还可以邀请新用户加入 OneDrive 你和受邀请者都可以获得0.5GB额外的储存空间,最高10GB),用户可选择付款扩展容量。OneDrive 现在支持100GB的文件。另外,微软也提供以 Silverlight 和 HTML5 为基础的文件上传功能,用户只要透过“拖拉”的方式便能将文件上传。但如没有安装 Silverlight 或浏览器不支持 HTML5 时,每次只能上传5个文件。

\subsection{使用 OneDrive}
打开\md{开始菜单},选择\md{所有应用},找到以 O 开头的 OneDrive 或者你可以点击字母快速索引找到 OneDrive 后单机打开,就会打开到如图 5.5 的资源管理器的 OneDrive 选项。
\begin{figure}[h]
	\begin{center}
		\includegraphics[width=\textwidth]{figure/prtsc/OD.png}
		\caption{OneDrive}
	\end{center}
\end{figure}
\subsection{从 OneDrive 打开和保存文件}
OneDrive 的工作方式与计算机上的所有其他文件夹类似。唯一的区别是 OneDrive 中的文件是在线同步的。因此,如果您想要保存任何文件,直接将它们放入 OneDrive 中。

\subsection{了解哪些文件位于 OneDrive 或你的电脑上}
您可以将文件保存在 OneDrive 上并使其在电脑上保持可用状态。

你可以将文件保存在您的电脑上,也可以在线备份。然后,你可以从电脑中删除它们。因为 \md{OneDrive 是一个云硬盘}。

\subsection{从互联网访问 OneDrive}
可以通过访问 onedrive.live.com 在互联网上访问 OneDrive。在这里你可以找到所有备份和保存的文件。

\section{章后练习}
\begin{reidai}
	请合理管理你的文件,可将重要文件备份到 OneDrive 以便不时之需,此外,可以访问\md{个人保管库}让你的文件更安全
\end{reidai}






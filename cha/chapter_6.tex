\chapter{使用程序、应用和文件}
\begin{center}
	\textcolor[RGB]{255, 0, 0}{\faHeart}我想成为一个温柔的人,因为曾被温柔的人那样对待,深深了解那种被温柔相待的感觉.\textcolor[RGB]{255, 0, 0}{\faHeart}
\end{center}
\rightline{——《夏目友人帐》}
\vspace{-5pt}
\begin{center}
	\pgfornament[width=0.36\linewidth,color=lsp]{88}
\end{center}

\section{任务栏基础}
你可以使用 Windows 11 搜索工具来查找可能难以找到的任何应用程序。单击开始菜单,在右上角的搜索框中输入你要查找的任何关键字,就会弹出该应用程序的一些相关建议,无论是在线还是在你的硬盘上。

通常,一些应用程序可以固定到开始菜单、桌面或任务栏,以便可以轻松找到它们。要将应用程序固定到任务栏,请打开应用程序并右键单击任务栏中正在运行的应用程序。然后选择固定到任务栏。

\begin{figure}[h]
	\begin{center}
		\includegraphics[width=0.5\textwidth]{figure/prtsc/pin.png}
		\caption{固定到任务栏}
	\end{center}
\end{figure}

\section{安装应用程序}
建议仅从受信任的站点购买或安装应用程序,因为这样可以减少系统感染病毒的机会。 Windows 11 的推荐平台是 Microsoft Store,从这里安装的软件还可以固定在开始菜单上。

\section{发送文档}
你可以发送文档到不同的驱动器和文件夹中,只需要右击(选中的)文件,显示更多选项,发送到:
\begin{figure}[h]
	\begin{center}
		\includegraphics[width=0.6\textwidth]{figure/prtsc/send_to.png}
		\caption{发送到}
	\end{center}
\end{figure}

\section{反人类的 Windows 11 菜单栏}
Win11 的新右键菜单相比之前的旧款式,颜值上的确大有提升,不仅使用了 Fluent Design 设计语言,而且优化了文字排版,行间距更宽,便于阅读和触控。然而,Win11 的右键菜单隐藏了很多选项,如果想要找到一些常用的功能,需要点击“显示更多选项”才能展开,这操作起来颇为麻烦。

\section{选择打开文件的程序}
默认情况下,文件将由系统中推荐的应用程序打开。但是,如果想使用其他程序或应用程序打开,只需找到该文件并右键单击它即可。然后在接下来出现的文本框中,选择\md{打开方式},如图 6.3:
\begin{figure}[h]
	\begin{center}
		\includegraphics[width=0.5\textwidth]{figure/prtsc/ways.png}
		\caption{打开方式}
	\end{center}
\end{figure}

\section{微软商店}
\subsection{位置}
 Microsoft 商店会出现在任务栏中,除非你将其从任务栏移除。不过,也可以简单地在开始菜单的搜索框中搜索它。
 
\subsection{从 Microsoft Store 应用添加新应用}
微软商店类似于手机的应用商店,可以从中获取应用,也可以购买游戏,例如我的世界:)

\section{卸载应用程序}
\subsection{卸载应用程序不是删除其快捷方式}
卸载应用程序这个操作看似很简单,实际上,很多人都没有做正确。大多数用户认为卸载软件的方法:在桌面上,右键点击应用,选择“删除”。

事实上,使用这一种方式是不能够把软件彻底删除掉的,只是删除了\md{快捷方式}而已,而这个软件还完好无损的在我们电脑里面。

在本书的 2.2.4 中,我对快捷方式下了个定义,快捷方式只是指向这个软件而已,而且,按照正常的情况,彻底删除了这个软件后,电脑磁盘也会释放这部分的空间,但是右键删除的方法,并不会给磁盘空间带来较大变化(一般快捷方式只有几 kb)。

\subsection{正确卸载软件}
如图 6.4 显示,前往所有应用,右击你要卸载的应用,点击卸载即可。
\begin{figure}[h]
	\begin{center}
		\includegraphics[width=0.6\textwidth]{figure/prtsc/uninstall.png}
		\caption{卸载软件}
	\end{center}
\end{figure}
另外,卸载软件的方法很多,这里只是一个我认为最简单的卸载方法,如果你想要列出所有软件(一般是你 C 盘满了删除一些或是寻找流氓软件),可以使用\md{控制面板}来卸载软件。

\section{学会操作文件路径}
当进行复制和剪切时,实际上是将文件从一个目录移动到另一个目录,在这个地方制作文件的副本。\md{复制}会保留源目录下的文件,使用快捷键 Ctrl+C 可以复制它。而\md{剪切}则不会保留源目录下的文件,通俗来讲可以比喻为移动该文件,使用快捷键 Ctrl+X 可以剪切它。决定好你是复制还是移动后,转到要创建副本的目录,使用 Ctrl+V 进行粘贴。

你可以通过单击鼠标一次来选择文件。但是,如果想选择多个文件,可以将鼠标放在屏幕的空白区域,按住后在要选择的文件上滑动。您也可以按住 Ctrl 按钮逐一选择文件,被选中的文件一般以深蓝色框选,如图 6.5:

\begin{figure}[h]
	\begin{center}
		\includegraphics[width=0.6\textwidth]{figure/prtsc/select.png}
		\caption{选择文件(夹)}
	\end{center}
\end{figure}
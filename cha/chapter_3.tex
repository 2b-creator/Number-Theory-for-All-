\chapter{桌面导航}
\begin{center}
	\textcolor[RGB]{255, 0, 0}{\faHeart}花有重开日,人无再少年.\textcolor[RGB]{255, 0, 0}{\faHeart}
\end{center}
\rightline{——《续侄溥赏酴醾劝酒二首其一》}
\vspace{-5pt}
\begin{center}
	\pgfornament[width=0.36\linewidth,color=lsp]{88}
\end{center}

\section{使用桌面}

上节我们说到可以直接从“开始”菜单启动应用程序,只需使用 Windows 键打开开始菜单,然后找到你的应用程序单击即可;不过桌面作为一个 Windows 从小到大的都有的组件,是我们打开程序的首选,我们从小上的电脑课,老师说,双击某个程序,你就可以打开它;这里的程序实则和开始菜单一样,也是一个\md{快捷方式}。

Windows 11 带来了添加\md{多个桌面的功能},其实这在早期的 10 也有了,不过这也意味着 Windows 11 将多任务处理提升到了一个不同的水平。

下面我从我们常用的一些方法讲起:

\subsection{更改桌面背景}
这方面其实很简单,找到一张图片,右击它,就可以设置为桌面背景。

或者你可以这么做:
\begin{ascolorbox9}{通过设置更改桌面背景}
	\begin{itemize}
		\item 在桌面右击后点击\md{个性化}
		\item 点击背景
		\item 选择通过更改图片或用纯色或幻灯片进行修饰来自定义背景。选择图片后,您可以浏览本地文件,找到适合您系统桌面的完美图片
		\item 何不试试 \md{Wall Paper Engine}
	\end{itemize}
\end{ascolorbox9}

\section{回收站}
每当使用完某个文件,不再想用它的时候,只需将它们放入回收站,它就会存储在那里,直到您再次需要它们为止。但是,如果一段时间过去了,那么它们将从系统中安全地删除。不过,您可以选择选择删除回收站中的文件的首选时间。

不过要注意的是,移动到回收站\md{不会增加你磁盘的剩余空间},因为它们仍旧保存在那里。

你可以在 Windows 11 中打开回收站,方法是转到桌面并单击回收站图标,默认情况下该图标位于屏幕的左上角。

但是,如果你在那里找不到它,请从开始菜单或使用快捷方式 Win+I 组合键打开设置。

\begin{ascolorbox9}{显示回收站及其他程序}
	\begin{itemize}
		\item 在设置面板左侧选择个性化,然后单击右侧中的主题
		\item 转到相关设置中的\md{桌面图标设置},并勾选\md{回收站},或者其他任何图标
		\item 完成后,点击\md{应用}后\md{确定}
	\end{itemize}
\end{ascolorbox9}

你还可以通过在 Windows 开始菜单搜索框中搜索来找到回收站。

\section{修改任务栏}
任务栏是 Windows 上最重要的东西之一。若你没有更改开始菜单的位置,这里左侧是小组件,中间是开始菜单,右侧则是后台程序运行的地方。和开始菜单一样,你可以把你的软件的快捷方式拖动到这里,或是打开的程序右击固定到任务栏,实现快捷访问。

但是,本人也不建议把所有软件放进来,例如手机的最下面你摆放的应用无非就是QQ,短信等你常用的东西,参照这些来定制你的任务栏。

\subsection{将所有窗口缩小到任务栏}
通过 Win+M 组合键,可以缩小所有已经打开的窗口到任务栏,然后通过 Win+Shift+M 可以将他们最大化。

此外 Win+↓ 可以一步步的缩小界面直到最小化,Win+↑ 则反之。

合理使用快捷键,可以增加你使用电脑的效率。

\subsection{从任务栏的跳转列表切换到不同的任务}
\begin{definition}{跳转列表}
	跳转列表指的是右击任务栏的应用程序以快捷操作最近文件等项目。如图3.1所示
\end{definition}
\begin{figure}[h]
	\begin{center}
		\includegraphics{figure/prtsc/jump_list.png}
		\caption{跳转列表}
	\end{center}
\end{figure}
使用 Windows 10 的人已经习惯了他们的跳转列表。他们所需要做的只是右键单击任务栏上的应用程序,你将看到您在该应用程序上所做的一切。然而,在 Windows 11 中,你右击应用程序,会发现啥都没有(在后续版本已修正)。

\begin{ascolorbox9}{让跳转列表显示更多信息}
	\begin{itemize}
		\item 在桌面右击后点击\md{个性化}
		\item 点击\md{开始}
		\item 选择在“开始”、“跳转列表”和“文件资源管理器”中显示最近打开的项目,将其设为开启
	\end{itemize}
\end{ascolorbox9}
这样,你就可以让跳转列表显示最近信息了。

\subsection{用好你的任务栏}
任务栏是 Windows 中的重要区域,它使工作变得更加轻松。

\section{操作中心和通知}
它位于任务栏的右侧,乍一看与 Windows 10 有相似之处;然而,仔细研究他们,你会发现他们根本不同。

\subsection{通知}
通知现在在日期和时间所在的位置弹出。然而,要查看通知界面,您需要单击日期和时间,当然这也会弹出日历。你可以在日历上方管理您的通知。

\subsection{快速设置图标}
\begin{definition}{快速设置}
	快速设置指的是 Windows 11 快捷菜单中的默认选项包括 Wi-Fi 连接、蓝牙连接、对焦辅助、飞行模式、辅助功能和投射。选择一个选项以将其打开或关闭。对于 Wi-Fi 按钮,它将连接到之前的 Wi-Fi 连接,同样,蓝牙图标将连接到任何可用的硬件。
\end{definition}

\begin{figure}[h]
	\begin{center}
		\includegraphics{figure/prtsc/quick_settings.png}
		\caption{快速设置}
	\end{center}
\end{figure}

为此,只需按 Win+A 快捷组合键,您的快速设置就会打开。此外,在操作中心,您可以打开 Wi-Fi 或启用一些易于访问的功能以及\md{切换蓝牙}或\md{夜间模式}。

\section{虚拟桌面}

\begin{definition}{虚拟桌面}
	在计算中,虚拟桌面是通过使用软件将计算机桌面环境的虚拟空间扩张到超出物理屏幕区域。其主要方式之一是可切换虚拟桌面,它允许用户制作他们的桌面视见区的虚拟副本,并在其间切换,具有着打开窗口存在于单一虚拟桌面之中。
\end{definition}
是不是被定义劝退了?哈哈哈,其实说人话就是再创建一个桌面,以此可以作为两个工作区,你可以在一个桌面干一件事情,然后再另一个桌面干另一个事情,互不干扰

\subsection{创建虚拟桌面}

虚拟桌面对于 Windows 来说并不是一个新功能了。在 Windows 10 中,创建一个单独的桌面是一件简单的事情,例如,您可以为一个项目使用一个桌面,为另一个项目使用第二个桌面,或者为您的工作使用一个桌面,为其他类型的东西使用第二个桌面。

不过,在 Windows 11 中,虚拟桌面有了升级。现在,您还可以为每个桌面设置不同的壁纸,从而更容易区分桌面(甚至可以为您提供不同的心情,具体取决于您使用它的用途)。其他一些功能使虚拟桌面的使用变得简单高效。请注意,其中一些是在 Windows 10 中引入的,但与 Windows 11 的新功能一起,它们构成了一个方便的工具箱。

\begin{ascolorbox9}{创建一个虚拟桌面}
	\begin{itemize}
		\item 将鼠标悬停在任务栏中的“\md{任务视图}”图标上或单击该图标(该图标看起来像是一个方块叠加在另一个方块上,如图3.3)
		\item 单击“\md{新建桌面}”缩略图。然后在任务栏上找到虚拟桌面,然后单击它切换
	\end{itemize}
\end{ascolorbox9}

\begin{figure}[h]
	\begin{center}
		\includegraphics{figure/prtsc/task_view.png}
		\caption{任务视图}
	\end{center}
\end{figure}

当然您也可以使用组合键 Win+Ctrl+D;在这种情况下,你会立即发现自己处于新桌面上。

您可以通过 Win+Ctrl+←(或→)组合键切换桌面。

您现在可以将不同的应用程序放置在单独的桌面上。单击“\md{任务视图}”图标可以从一个桌面移动到另一个桌面。(您还可以使用熟悉的 \md{Alt+Tab} 组合键四处移动,这将带您转到一个桌面上的所有应用程序,然后转到下一个桌面上的应用程序。)

\subsection{移除虚拟桌面}
悬停在\md{任务视图}图标上后删除一个桌面。

\section{章后练习}
\begin{reidai}
	请善用本章的知识,改进一下你使用 Windows 11的习惯,好好思考一下你应该如何更好的使用这个操作系统。
\end{reidai}